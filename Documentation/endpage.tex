%%%%%%%%%% Insert bibliography here %%%%%%%%%%%%%%

% \section*{Bibliographie}
\footnotesize{
 
% \noindent [1] Bacro, J. N., \& Gaetan, C. (2012), {\it A review on spatial extreme modelling. In Advances and Challenges in Space-time Modelling of Natural Events}. Springer Berlin Heidelberg, 103--124.

% \noindent [2] Beuvier, J., Sevault, \ldots \& Somot, S. (2010), {\it Modeling the Mediterranean Sea interannual variability during 1961–2000: focus on the Eastern Mediterranean Transient}, Journal of Geophysical Research: Oceans (1978–2012), 115(C8).

% \noindent [3] Blanchet, J., \& Davison, A. C. (2011), {\it Spatial modeling of extreme snow depth}, The Annals of Applied Statistics, 5(3), 1699-1725.

% \noindent [4] Davison, A. C., \& Gholamrezaee, M. M. (2012), {\it Geostatistics of extremes}, Proceedings of the Royal Society A: Mathematical, Physical and Engineering Science, 468(2138), 581-608.

% \noindent [5] De Haan, L. (1984), {\it A spectral representation for max-stable processes}, The Annals of Probability, 12(4), 1194-1204.

% \noindent [6] Fisher, R. A., \& Tippett, L. H. C. (1928), {\it Limiting forms of the frequency distribution of the largest or smallest member of a sample}, In Mathematical Proceedings of the Cambridge Philosophical Society (Vol. 24, No. 02, pp. 180-190). Cambridge University Press.

% \noindent [7] Gnedenko, B. (1943), {\it Sur la distribution limite du terme maximum d'une serie aleatoire}, Annals of Mathematics, 423-453.

% \noindent [8] Herrmann, M. J., \& Somot, S. (2008), {\it Relevance of ERA40 dynamical downscaling for modeling deep convection in the Mediterranean Sea}, Geophysical Research Letters, 35(4).

% \noindent [9] Kabluchko, Z., Schlather, M., \& De Haan, L. (2009), {\it Stationary max-stable fields associated to negative definite functions}, The Annals of Probability, 2042-2065.

% \noindent [10] Miller, J., \& Dean, R. (2004), {\it A simple new shoreline change model}, Coastal Engineering, 51(7), 531-556.

% \noindent [11] Ministère de l'Ecologie, du Developpement Durable et de l'Energie (2012), {\it Stratégie nationale de gestion intégrée du trait de côte : vers la relocalisation des activités et des biens}, Rapport Ministériel, repéré à \url{http://www.developpement-durable.gouv.fr/Strategie-nationale-de-gestion.html}.

% \noindent [12] Raillard, N., Ailliot, P., \& Yao, J. F. (2013), Modeling extreme values of processes observed at irregular time steps: application to significant wave height.

% \noindent [13] Resnick, S. I. (1987), {\it Extreme Values, Regular Variation, and Point Processes}.

% \noindent [14] Ribatet, M. (2011), {\it SpatialExtremes: modelling spatial extremes}, R package version, 1-8.

% \noindent [15] Schlather, M. (2002), {\it Models for stationary max-stable random fields}, Extremes, 5(1), 33-44.

% \noindent [16] Smith, R. L. (1990), {\it Max-stable processes and spatial extremes}, Unpublished manuscript, Univer.

% \noindent [17] Tolman, H. L. (2014), {\it User manual and system documentation of WAVEWATCH III TM version 4.18}, Technical note, NOAA/NWS/NCEP/MMAB Contribution, (316).
}

\end{document}
